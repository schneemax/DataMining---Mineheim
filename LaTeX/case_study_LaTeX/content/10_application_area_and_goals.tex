\chapter{Introduction \& Procedure}
\section{Application area and goals}
%Business Understanding

% {– What is the actual problem (in the domain)?
% – What is the target variable?
% – Clustering/Classification/Regression?

This paper represents the project report of Team MINEheim (Group 1) and documents the problem, purpose, approach, and the results of the programming project which was completed in parallel. To begin with, the underlying data mining problem that is addressed by this paper shall be described.

As mentioned in the proposal document, this project is related to the wine industry, especially the quality of wine. The main problem that needs to be solved is the fact that the quality of wine can not be objectively determined. In addition, every existing approach to rate a wine's quality includes its actual consumption which is not acceptable for rare or highly-expensive products. Wine sommeliers try to solve this problem by communicating their rating of the specified wine to the end consumer, but their taste is rather subjective and therefore not reliable. In fact, this constitutes a weak spot of the wine industry since it heavily relies on the subjective taste of a few sommeliers. As a result, there is no single source of truth about the ingredients, components or features of wine that influence its quality. In short, the main question is: What makes a good wine actually good?

Consequently, the goal of this project was to objectively predict and classify wine based on its ingredients, components, and features to enable an objective wine rating. The target variable was the quality of wine which originally consisted of values on a scale from zero to ten. Later, the target variable was categorized into three distinct classes: (class 0: low, class 1: medium, class 2: high). Due to the categorization of the target variable into classes, this project deals with a classification approach of machine learning. In theory, classification is one of the most common tasks of supervised machine learning. It describes the classification of unknown instances of input data in one of the pre-offered categories, the classes, based on a previously trained model. \citep{Novakovic2010}

The solution which was developed during this project aims to leverage various benefits for businesses and consumers. In detail, those benefits were already addressed in the proposal document as they range from cost and time savings to an objective single source of truth and many others. 