\section{Structure and Size of the Dataset}\label{sec:data_structure}
%Data understanding

% – Are examples sorted (time series)?
% – What is the distribution of labels / target variable?
% – Are all attributes and their types listed?
% – Are attributes explained?
% – What is the quality of the data?}

The dataset used in this project is related to red and white variants of the Portuguese wine \textit{"Vinho Verde"}, a unique wine product from the northwest region of Portugal. The data was originally collected by an inter-professional organization called CVRVV between 2004 and 2007 for marketing and quality assurance purposes \citep{misc_wine_quality_186}. The whole set includes 6,497 entries.

Taking a closer look at the data, 13 unique features can be identified, which describe each wine sample. Most of the features represent test results of physiochemical properties of each specific wine i.e., \textit{fixed acidity, volatile acidity, citric acidity, residual sugar, chlorides, free sulfur dioxide, total sulfur dioxide, density, pH-value, sulphates,}  and \textit{alcohol}. These attributes are given in their respective measurement unit and represented as positive numeric values. The remaining features comprise the \textit{quality} and \textit{type} of the wine. The quality of a wine is represented by a score metric ranging from zero to ten, where zero is a wine of low quality and ten a wine of excellent quality. Each wine was evaluated with a sensory test (blind test) by at least three assessors and then graded. The final grade was calculated with the median of these tests. The feature \textit{type} indicates whether the wine is a red or a white wine.

When searching for null or missing values only 34 of all samples had missing values. Concerning the value distributions of the features, it can be noticed that some of the features e.g., \textit{alcohol}, have a slightly right-skewed distribution. Although some outliers can be found, after some investigation it turned out to be no issue for the use of the data in this project. Looking at the \textit{type} feature it becomes apparent that there are about twice as many white wines than red wines. What can also be noticed and is a really important insight for the preprocessing is that the \textit{quality} feature is highly imbalanced. The quality values five and six together make up about 76\%, or 5,000 samples, of the whole dataset. This means that there are a lot of wines with medium quality and few of low and high quality. This problem has to be addressed in the preprocessing step.
\pagebreak