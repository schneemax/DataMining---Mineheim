\section{Structure and size of the data set}\label{sec:data_structure}
%Data understanding

% – Are examples sorted (time series)?
% – What is the distribution of labels / target variable?
% – Are all attributes and their types listed?
% – Are attributes explained?
% – What is the quality of the data?}

The data set used in this project is related to red and white variants of the Portuguese wine "Vinho Verde", a unique wine product from the northwest region of Portugal. The data was originally collected by an inter-professional organization called CVRVV between 2004 and 2007 for marketing and quality assurance purposes (LINK DATA SET?). The whole set includes 6497 entries.
Taking a closer look at the data, 13 unique features can be identified which describe each wine sample. Most of the features represent test results of physiochemical properties of each specific wine e.g., fixed acidity, volatile acidity, citric acidity, residual sugar, chlorides, free sulfur dioxide, total sulfur dioxide, density, pH-value, sulphates, alcohol. The remaining features comprise the quality and type of the wine. Concerning the distributions of the features, it can noticed that the quality feature is highly imbalanced. There are a lot of wines with medium quality and less of low and high quality. Also some of the features have a positive skewness