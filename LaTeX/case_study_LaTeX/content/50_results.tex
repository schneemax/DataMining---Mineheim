\chapter{Results}
% Result
% – Is the result critically evaluated?
% – Is the result compared against the baseline?
% – What does the result mean w.r.t. the problem? (i.e., could you use it?)
To conclude, our business use case was to predict the quality of wine with the help of the contents. In order to do that we used a dataset consisting of 6,497 samples of wines with their respective physiochemical features and their quality which was assessed by sommeliers. The data was unbalanced in respect to the distribution over the classes. To combat this problem and to prepare the data to be used in the algorithms, we performed a set of preprocessing steps e.g., normalization and balancing. After having preprocessed the data we could start to test out different algorithms. We used five algorithms to try to predict wine quality. 

KNN as our baseline did good job in regards to its simplicity. The model was overfitted in the end but nevertheless produced test scores of 60\%. The decision tree has a mediocre performance on its own. It also was overfitted when trying oversampling and feature selection but produced a test score of 57.5\% - although the best score was reached without any preprocessing and the default values. Random forest performed quite well as an aggregation of decision trees. Like all of our models, the random forest had a tendency to overfitting but in the end it reached the overall best score with all preprocessing steps of 70.9\%. The support vector machine also performed reasonable as a classification algorithm in our case with the problem of being slightly overfitted but reaching a score of 57.6\%.
Although the predictions of the neural network were okay with 56.7\% at its highest, the efforts to create the network with different architectures did not pay off. We would need more resources and better experience with neural networks to improve the score in this regard. To conclude, all of our models had a tendency to overfitting.
Of our five models the random forest algorithm did the best job, while also being the only model which performed better than the baseline algorithm.
Nevertheless, all of our models could not with a high certainty predict the quality of the wines on the basis of its contents. This is probably due to the fact that also other features besides chemical contents determine how someone perceives the quality of a particular wine.