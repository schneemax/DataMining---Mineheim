% do not change these two lines (this is a hard requirement
% there is one exception: you might replace oneside by twoside in case you deliver 
% the printed version in the accordant format
\documentclass[11pt,titlepage,oneside,openany]{article}
\usepackage{times}


\usepackage{graphicx}
\usepackage{latexsym}
\usepackage{amsmath}
\usepackage{amssymb}

\usepackage{ntheorem}

% \usepackage{paralist}
\usepackage{tabularx}

% this packaes are useful for nice algorithms
\usepackage{algorithm}
\usepackage{algorithmic}

%Packages Max Darmstadt
\usepackage{lipsum}
\usepackage{xargs}%For todo notes
\usepackage[colorinlistoftodos,prependcaption,textsize=tiny]{todonotes} % Commands for more todo options
\newcommandx{\unsure}[2][1=]{\todo[linecolor=red,backgroundcolor=red!25,bordercolor=red,#1]{#2}}
\newcommandx{\change}[2][1=]{\todo[linecolor=blue,backgroundcolor=blue!25,bordercolor=blue,#1]{#2}}
\newcommandx{\info}[2][1=]{\todo[linecolor=green,backgroundcolor=green!25,bordercolor=green,#1]{#2}}
\newcommandx{\improvement}[2][1=]{\todo[linecolor=purple,backgroundcolor=purple!25,bordercolor=purple,#1]{#2}}
\newcommandx{\thiswillnotshow}[2][1=]{\todo[disable,#1]{#2}} % usable todo commands: todo, unsure, change, info, improvement, thiswillnotshow

%Packages Nicolas Fürhaupter
\usepackage{color}

% well, when your work is concerned with definitions, proposition and so on, we suggest this
% feel free to add Corrolary, Theorem or whatever you need
\newtheorem{definition}{Definition}
\newtheorem{proposition}{Proposition}


% its always useful to have some shortcuts (some are specific for algorithms
% if you do not like your formating you can change it here (instead of scanning through the whole text)
\renewcommand{\algorithmiccomment}[1]{\ensuremath{\rhd} \textit{#1}}
\def\MYCALL#1#2{{\small\textsc{#1}}(\textup{#2})}
\def\MYSET#1{\scshape{#1}}
\def\MYAND{\textbf{ and }}
\def\MYOR{\textbf{ or }}
\def\MYNOT{\textbf{ not }}
\def\MYTHROW{\textbf{ throw }}
\def\MYBREAK{\textbf{break }}
\def\MYEXCEPT#1{\scshape{#1}}
\def\MYTO{\textbf{ to }}
\def\MYNIL{\textsc{Nil}}
\def\MYUNKNOWN{ unknown }
% simple stuff (not all of this is used in this examples thesis
\def\INT{{\mathcal I}} % interpretation
\def\ONT{{\mathcal O}} % ontology
\def\SEM{{\mathcal S}} % alignment semantic
\def\ALI{{\mathcal A}} % alignment
\def\USE{{\mathcal U}} % set of unsatisfiable entities
\def\CON{{\mathcal C}} % conflict set
\def\DIA{\Delta} % diagnosis
% mups and mips
\def\MUP{{\mathcal M}} % ontology
\def\MIP{{\mathcal M}} % ontology
% distributed and local entities
\newcommand{\cc}[2]{\mathit{#1}\hspace{-1pt} \# \hspace{-1pt} \mathit{#2}}
\newcommand{\cx}[1]{\mathit{#1}}
% complex stuff
\def\MER#1#2#3#4{#1 \cup_{#3}^{#2} #4} % merged ontology
\def\MUPALL#1#2#3#4#5{\textit{MUPS}_{#1}\left(#2, #3, #4, #5\right)} % the set of all mups for some concept
\def\MIPALL#1#2{\textit{MIPS}_{#1}\left(#2\right)} % the set of all mips


\begin{document}

\pagenumbering{roman}
% lets go for the title page, something like this should be okay
\begin{titlepage}
	\vspace*{2cm}
  \begin{center}
   {\Large Outline Proposal Paper\\}
   \vspace{2cm} 
   {Data Mining 1\\}
   \vspace{2cm}
   {presented by\\
    Anna-Lena Blinken (1818326)\\
	Nicolas Hautschek (1816720)\\
	Max Darmstadt (1820000)\\
	Erik Penther (1602026)\\
	Nicolas Fürhaupter (1819446)\\
   }
   \vspace{1cm} 
   %{submitted to the\\
   % Data and Web Science Group\\
   % Prof.\ Dr.\ Right Name Here\\
   % University of Mannheim\\} \vspace{2cm}
   {HWS 2021}
  \end{center}
\end{titlepage} 

% no lets make some add some table of contents
\tableofcontents
\newpage

%\listofalgorithms

%\listoffigures

%\listoftables

% evntuelly you might add something like this
% \listtheorems{definition}
% \listtheorems{proposition}

\newpage


% okay, start new numbering ... here is where it really starts
\pagenumbering{arabic}

\section{What is the problem you are solving?}

This document represents the case study proposal for Team MINEheim (Group 1). After extensive research in different topic areas and comparing multiple datasets that might be relevant for such a student project, we decided that we’d like to focus on the wine industry. This is also due to the fact that every group member is interested in research questions regarding the mentioned industry and we definitely see a lot of potential for different data mining techniques to analyze our dataset.

With the help of our dataset and various data mining methods, we want to objectively classify wine based on its distinct attributes. To go into more detail, we want to predict the quality of wine based on its components and features. The target variable is the wine quality. Our approach solves the following problem and leads to a valuable business case for target groups:

First of all, the prediction of wine quality based on its components is highly-valuable for wineries and large wine dealers. By following our approach, wineries and dealers can objectively determine the quality of their products which could lead to a higher market success. Moreover, our case study facilitates the valuing of (old) wine without actually drinking it. In general, you would use experienced wine sommeliers to determine the quality of wine, but those are rare, require long training and increase the costs for product development. It is also worth mentioning that due food regulations, wine ingredients are measured anyways. By predicting the wine quality with data mining algorithms, it’s possible to shorten the taste-testing procedure of experienced sommeliers and to save important financial resources. Our solution enables possible customers to recommend their wine to consumers and solves the weak spot that the whole wine industry heavily relies on the subjective taste of a few wine sommeliers.

\section{What data will you use?}

\section{How will you solve the problem?}\unsure{Müssen wir das kürzen?}
\subsection{What preprocessing steps will be required?}
To prepare the data we received in the dataset to be useful for our different algorithms a couple of preprocessing
steps will be required. This includes the following:

\textbf{Under-/oversampling -- data balancing} In a first glance at the data we have noticed that the class distribution is
unbalanced (there are more normal wines than great or bad ones). For that reason we have to decide what to do about it -- how many classes we want to predict (e.g. binary classification or three classes: bad, normal and good), and how we want to distribute the samples among the classes so that it is balanced (e.g. via under- or oversampling).

\textbf{Data analysis and inspection} The next step would be to have a look at the data whether there are duplicates, null values or outliers in the data. Additionally, we can analyse whether there is any correlation between different features in the dataset. Furthermore, we could have a look into if we want to indroduce additional features to improve the classifier.

\textbf{Feature standardization of numerical values (normilaztion)} In order to have good results when building the classifiers we have to bring each numerical features into the same scale so one feature doesn't have a bigger impact on the outcome than others. For that reason we have do standardize or normalize the numerical features.

\subsection{Which algorithms do you plan to use?}
We plan on using six to seven different classifiers to solve the problem. Since we 
are trying to solve a classification task, the algorithms we plan on using are classification
algorithms.

Firstly, we are planning on using \textbf{k-nearest neighbors (k-NN)} as a baseline\unsure{Passt das mit der Baseline?}
algorithm for our classification problem as k-NN is quite a simple algorithm.

Additionally, we want to use \textbf{decision trees} to try build a model to classify the
data. As an extension of decision trees we are also interested in using a \textbf{random forest}
to derive a classifier.

Furthermore, other classification algorithms that we want to use include a \textbf{Support Vector Machine (SVM)}
or \textbf{logistic regression} -- although logistic regression could only be used if we decide on
a binary classification problem (i.e. deciding between wines of good and bad quality).

Moreover, we want to build a \textbf{neural network} that can predict if a given wine is of good or bad quality.

Finally, we are interested in using \textbf{outlier detection / anomaly detection algorithms} to detect the few 
excellent or poor wines that are given in the dataset.

Generally we want to use \textbf{feature selection} test for each algorithm if there are any features which are not relevent 
for classifying between good and bad wines to exclude them from the model. In addition to that we want to use 
\textbf{hyperparameter tuning} for the different algorithms to optimize the models.

\section{How will you measure success?}

\section{What do you expect your results to look like?}
We generally expect that there is a correlation between the make up of different wines and their taste (quality). Without analyzig the particular data set we found different sources that link for example the ph-value of wine to whether it is sour or mild. Furthermore attributes like alcohol percentage or sugar effect the sweetness of wine. Even though taste and quality might be subjective, for example sweeter wine might be generally more liked by a broader audience than bitter wine. That's why we expect to being able to predict the quality of different wines. Nonetheless the precision of our model could be rather poor, just because of the fact that quality and taste are subjective. On the other hand the quality of wine must be influenced by its ingredients, because of that our results might be more precise than we initially expect.

%%% Un-Comment to use the bibliography
%\bibliographystyle{plain}
%\bibliography{thesis-ref.bib}


%\appendix


%\newpage


%\pagestyle{empty}


%\section*{Ehrenw\"ortliche Erkl\"arung}
%Ich versichere, dass ich die beiliegende Master-/Bachelorarbeit ohne Hilfe Dritter
%und ohne Benutzung anderer als der angegebenen Quellen und Hilfsmittel
%angefertigt und die den benutzten Quellen w\"ortlich oder inhaltlich
%entnommenen Stellen als solche kenntlich gemacht habe. Diese Arbeit
%hat in gleicher oder \"ahnlicher Form noch keiner Pr\"ufungsbeh\"orde
%vorgelegen. Ich bin mir bewusst, dass eine falsche Er- kl\"arung rechtliche Folgen haben
%wird.
%\\
%\\

%\noindent
%Mannheim, den 31.08.2014 \hspace{4cm} Unterschrift

\end{document}
